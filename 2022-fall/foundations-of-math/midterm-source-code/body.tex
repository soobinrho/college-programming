
% ----------------------------- %
% Section 1
% ----------------------------- %
\section*{Proof}

In order to prove this generalization,
the first thing I did was to translate it into more
primitive terms by using
the sentence-form definition of rational numbers.

"If $x$ is rational and $y$ is rational,
then $x+y$ is rational" is equivalent to
\begin{equation} \label{eq:1}
	\forall \;
	x,y \in \mathbb{R}, \; \;
	\big( x \in Q \; \text{ and } \; y \in Q \big)
	\Rightarrow
	(x+y) \in Q
\end{equation}

In order to prove this,
\begin{equation}
	\text{Let} \; x,y \in Q
\end{equation}

Using the sentence-form definition of rational
numbers, we get
\begin{align} 
	\text{Then,} \;\; &\exists \; p,q \in \mathbb{Z}
	\;\; \text{s.t.} \;\; x=\frac{p}{q}
	\; \text{and} \; q \neq 0 \label{eq:q}
	\\
	&\exists \; m,n \in \mathbb{Z}
	\;\; \text{s.t.} \;\; y=\frac{m}{n}
	\; \text{and} \; n \neq 0 \label{eq:n}
\end{align}

What I'm trying to do here is to show that
both the hypothesis and the conclusion are
true for all $x,y \in \mathbb{R}$.
Proving that this conditional
sentence is always true proves that it is
a true generalization.
\begin{align}
	\text{Then,} \;\; (x + y) &= \frac{p}{q} + \frac{m}{n}
	\\
	&= \frac{p n + q m}{q n} 
\end{align}

From here, how can I know if 
$(x + y) = \frac{p n + q m}{q n}$ is a rational number?
Just to remind myself, here's what it means to be
a rational number: 

$x$ is a rational number iff
$\exists \; a,b \in \mathbb{Z}
\;\; s.t. \;\; x = \frac{a}{b} 
\; \text{and} \; b \neq 0$

So, in this case, what I had to do was to
show that both the numerator $p n + q m$
and the denominator $q n$ as in
$(x + y) = \frac{p n + q m}{q n}$
are integers.

How can I know whether or not
$p n + q m$ and $q n$ are integers?
I used Prior Results given by Dr. Perry.
First,
\textit{Prior Result 0.2} states that
integers are closed under addition.
This means that an addition of integers
results in another integer. So, $p n + q m$
results in an integer. Likewise,
\textit{Prior Result 0.4} states that
integers are closed under multiplication.
So, $q n$ results in an integer, too.
Going back to the original expression,
we now know that
$(x + y) = \frac{p n + q m}{q n}$
always results in an integer over integer.

One more thing. In order for
$(x + y) = \frac{p n + q m}{q n}$
to be a rational number,
both the numerator and denominator
have to be integers, which we just
proved. However, there's one more condition.
The denominator $q n \neq 0$.
This, however, is already true because
we defined $q \neq 0$ in \eqref{eq:q} and
$n \neq 0$ in \eqref{eq:n}.
\begin{align}
	&\text{Since} \; (x + y) = \frac{p n + q m}{q n}
	\;\; \text{and} \;\; 
	p,q \in \mathbb{Z}
	\;\; \text{and} \;\;
	m,n \in \mathbb{Z}
	\;\; \text{and} \;\;
	q n \neq 0,
	\\
	&\exists \; a = (p n + q m)
	\in \mathbb{Z}, \;
	b = (q n)
	\in \mathbb{Z}
	\;\; s.t. \;\; (x + y) = \frac{a}{b} 
	\; \text{and} \; b \neq 0
	\\
	&\therefore \;
	\forall \;
	x,y \in \mathbb{R} \text{,} \; \;
	( x \in Q \; \text{ and } \; y \in Q )
	\Rightarrow
	(x+y) \in Q	
\end{align}

\newpage

% ----------------------------- %
% Section 2
% ----------------------------- %
\bgroup
\color{abs}
\hrule
\egroup

\begin{abstract}
(b) Select a false conjecture from the list and
provide a counterexample.

\: \; \textbf{
D. If $x$ is irrational and $y$ is irrational,
then $x + y$ is rational.
}
\noindent\textbf{\\}
\end{abstract}

\bgroup
\color{abs}
\hrule
\egroup

\section*{Counterexample}

First, translate.
\begin{equation}
	\forall \;
	x,y \in \mathbb{R}, \; \;
	\big[ (x \notin Q \; \text{ and } \; y \notin Q)
	\Rightarrow
	(x+y) \in Q \big]
\end{equation}

Negation of a conditional sentence
is an existment statement with the hypothesis
being true and the conclusion being false.
\begin{equation}
	\exists \;
	x,y \in \mathbb{R} \;\; \text{s.t.} \;\;
	\big[ (x \notin Q \; \text{ and } \; y \notin Q)
	\; \text{and} \;
	(x+y) \notin Q \big]
\end{equation}

Any value of $x,y \in \mathbb{R}$
that makes this negation true will suffice.
I'll go with $\pi$.
\begin{align}
	&\text{Let} \;
	x = \pi, \; y = \pi,
	\\
	&\text{Then,} \;
	x = \pi \notin Q \; \text{and} \;
	y = \pi \notin Q, \; \text{but}
	\\
	&(x + y) = \pi + \pi = 2 \pi \notin Q
	\\
	&\therefore \;
	\exists \;
	x,y \in \mathbb{R} \;\; \text{s.t.} \;\;
	\big[ (x \notin Q \; \text{ and } \; y \notin Q)
	\; \text{and} \;
	(x+y) \notin Q \big]
\end{align}

The question of "how can we prove if $\pi$
is an irrational number?" still remains,
but it seems like our class will be covering
this topic in the \textit{Chapter 3.4}.

\newpage

% ----------------------------- %
% Section 3
% ----------------------------- %
\bgroup
\color{abs}
\hrule
\egroup

\begin{abstract}
	(c) Select a true conjecture from the list and
	give a proof for why it is true.
	
	\: \; \textbf{
		E. If $x$ is rational and $x + y$ is irrational,
		then $y$ is irrational.
	}
	\noindent\textbf{\\}
\end{abstract}

\bgroup
\color{abs}
\hrule
\egroup

\section*{Proof}

Again, translation first.
\begin{equation}
	\forall \;
	x,y \in \mathbb{R}, \; \;
	\big( x \in Q \; \text{ and } \; (x + y) \notin Q \big)
	\Rightarrow
	y \notin Q
\end{equation}

At first glance, it's hard to know
how to prove this, but we can use
logical equivalence to make this
easier to understand.
Using \textit{A Version of the Contrapositive},
we get
\begin{align}
	&(A \; \text{and} \; B) \Rightarrow C
	\\
	\text{LE to} \;\;
	&(A \; \text{and} \; \text{not}(C)) \Rightarrow
	\text{not}(B)
\end{align}
\begin{align}
	&\forall \;
	x,y \in \mathbb{R}, \; \;
	\big( x \in Q \; \text{ and } \; (x + y) \notin Q \big)
	\Rightarrow
	y \notin Q
	\\
	\text{LE to} \;\;
	&\forall \;
	x,y \in \mathbb{R}, \; \;
	\big( x \in Q \; \text{ and } \; y \in Q \big)
	\Rightarrow
	(x + y) \in Q
\end{align}

This generalization therefore happens to be
the same as the generalization \eqref{eq:1}
from the first question. Beautiful!
\begin{equation*}
	\forall \;
	x,y \in \mathbb{R}, \; \;
	\big( x \in Q \; \text{ and } \; y \in Q \big)
	\Rightarrow
	(x+y) \in Q
\end{equation*}

Thanks to the logical equivalence,
we now know that this question is acutally
the same as the first question.
The same proof for the first question
can be used for this question.
\begin{align}
	\text{Let} \;\;
	&x \in Q \; \text{and} \; y \in Q,
	\\
	\text{Then,} \;\;
	&\exists \; p,q \in \mathbb{Z}
	\;\; \text{s.t.} \;\; x=\frac{p}{q}
	\; \text{and} \; q \neq 0
	\\
	&\exists \; m,n \in \mathbb{Z}
	\;\; \text{s.t.} \;\; y=\frac{m}{n}
	\; \text{and} \; n \neq 0
	\\
	&(x + y) = \frac{p n + q m}{q n}
	\;\; \text{and} \;\; 
	p,q \in \mathbb{Z}
	\;\; \text{and} \;\;
	m,n \in \mathbb{Z}
	\;\; \text{and} \;\;
	q n \neq 0
	\\
	&\exists \; a = (p n + q m)
	\in \mathbb{Z}, \;
	b = (q n)
	\in \mathbb{Z}
	\;\; s.t. \;\; (x + y) = \frac{a}{b} 
	\; \text{and} \; b \neq 0
	\\
	&\therefore \;
	\forall \;
	x,y \in \mathbb{R} \text{,} \; \;
	( x \in Q \; \text{ and } \; y \in Q )
	\Rightarrow
	(x+y) \in Q	
\end{align}