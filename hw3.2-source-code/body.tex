
% ----------------------------- %
% Section 1
% ----------------------------- %
\section*{Proof}

The first thought that came to my mind was, can I use
Theorem 1.6.1 (Transitivity) ``If $A \Rightarrow B$ and 
$B \Rightarrow C$, then $A \Rightarrow C$'' to solve
this problem? Maybe I could, but then I couldn't find
a way to translate our original problem into this this
theorem. So, I took a different approach.
\begin{flalign*} 
	& & &\text{Assume } a < b \text{ and } b < c \text{,} & &
	\\
	& & &\text{Then, } a < b  & & \llap{by assumption}
	\\
	& & &\Leftrightarrow  a - a  < b - a & & \llap{Theorem 4}
	\\
	& & &\Leftrightarrow  0  < b - a & & \llap{Axiom 1}
	\\
	& & &\text{Also, } b < c  & & \llap{by assumption}
	\\
	& & &\Leftrightarrow  b - b  < c - b & & \llap{Theorem 4}
	\\
	& & &\Leftrightarrow  0  < c - b & & \llap{Axiom 1}		
	\\
	& & &\text{Since the sum of positive numbers is positive,} & &\llap{by Axiom 1B}
	\\
	& & &(b - a) + (c - b) > 0
	\\
	& & &\Leftrightarrow -a + c > 0
	\\
	& & &\Leftrightarrow -a > -c
	\\
	& & &\Leftrightarrow a < c \;\;\; \qedsymbol & & \llap{Theorem 7}
\end{flalign*}

\hfill
\newpage

% ----------------------------- %
% Section 2
% ----------------------------- %
\bgroup
\color{abs}
\hrule
\egroup

\begin{abstract}
	B11. Reproduce and complete the given parts of the proof of
	Theorem 8F, and finish the proof:
	
	\textbf{
		\\
		If $a \neq 0$, then $a^{2} > 0$.
	}
	\noindent\textbf{\\}
\end{abstract}

\bgroup
\color{abs}
\hrule
\egroup

\section*{Proof}

\begin{flalign*} 
	& & &\text{If }a\neq0\text{, then }a\text{ is positive or}-a\text{ is positive.} & & \llap{Axiom 1}
\end{flalign*}
This is in the form of $H \Rightarrow (A$ or $B)$.
So, I'll prove this by proving both cases -- i.e.
both the case in which $a$ is positive and the other
case in which $-a$ is positive.
\begin{flalign*} 
	& & &\text{If }a\text{ is positive, then }a>0 & & \llap{Axiom 1A}
	\\
	& & &\text{Also, since }a^{2} = a \cdot a \text{,} & & \llap{Prior Result 0}
	\\
	& & &\text{and since a product of two positive numbers is positive,} & & \llap{Prior Result 0}
	\\
	& & &a^{2} > 0 & & \llap{Theorem 7}
	\\
	\\
	& & &\text{If}-a\text{ is positive, then}-a>0 & & \llap{Axiom 1A}
	\\
	& & &\text{Also, since }a^{2} = a \cdot a \text{,} & & \llap{Prior Result 0}
	\\
	& & &\text{and since a product of two negative numbers is positive,} & & \llap{Prior Result 0}
	\\
	& & &a^{2} > 0 \;\;\; \qedsymbol & & \llap{Theorem 7}	
\end{flalign*}
Reflection on myself: I feel guilty about just assuming that
a product of (two positive numbers and two negative numbers) 
is positive becaues of Prior Result 0.
Is there a better way of proving this?

Update: I just found out that Axiom 1C actually states that
the product of positive numbers is positive!

\newpage

% ----------------------------- %
% Section 3
% ----------------------------- %
\bgroup
\color{abs}
\hrule
\egroup

\begin{abstract}
	B13. Prove Theorem 13B:
	
	\textbf{
		\\
		(Multiplying the Sides of an Inequality by a Number)
		\\
		If $c < 0$ and $a < b$, then $ca > cb$.
	}
	\noindent\textbf{\\}
\end{abstract}

\bgroup
\color{abs}
\hrule
\egroup

\section*{Proof}

\begin{flalign*} 
	& & &\text{Assume }c<0\text{ and }a<b\text{,} & & \llap{}
	\\
	& & &\text{Then, }a<b\text{ implies }0<b-a & & \llap{Theorem 4}
	\\
	& & &\text{Therefore, }b-a\text{ is positive.} & & \llap{Axiom 1A}
	\\
	& & &\text{Also since}-c\text{ is positive, } & & \llap{by assumption}
	\\
	& & &0<(b-a)(-c) & & \llap{Axiom 1C}
		\\
	& & &\Leftrightarrow 0<-cb+ca & & \llap{Prior Result 0}
	\\
	& & &\Leftrightarrow -ca<-cb & & \llap{Theorem 4}
	\\
	& & &\Leftrightarrow ca>cb \;\;\; \qedsymbol & & \llap{Theorem 7}
\end{flalign*}

\hfill
\newpage

% ----------------------------- %
% Section 4
% ----------------------------- %
\bgroup
\color{abs}
\hrule
\egroup

\begin{abstract}
	B20. Resolve Conjecture 20:
	
	\textbf{
		\\
		If $c \geq 1$, then $cx \geq x$.
	}
	\noindent\textbf{\\}
\end{abstract}

\bgroup
\color{abs}
\hrule
\egroup

\section*{Counterexample}

My intuition tells me that this conjecture is false.
Thus, I'll come up with a counterexample, where
the conjecture's negation is true.
\begin{flalign*} 
	& & &\text{Let }c=2, x=-20, & & \llap{}
	\\
	& & &\text{Then }c=2\geq 1 \text{, but }cx=(2)(-20)=-40<-20=x & & \llap{}
\end{flalign*}

\hfill
\newpage

% ----------------------------- %
% Section 5
% ----------------------------- %
\bgroup
\color{abs}
\hrule
\egroup

\begin{abstract}
	B24. Fix Conjecture 20 and provide a proof of it:
	
	\textbf{
		\\
		Conjecture 20:
		If $c \geq 1$, then $cx \geq x$.
	}
	\noindent\textbf{\\}
\end{abstract}

\bgroup
\color{abs}
\hrule
\egroup

\section*{Proof}

Conjecture 20 is a false generalization, but it can
become a true generalization if we put one more condition
into the hypothesis: 
\begin{equation*}
	\text{If }c \geq 1\text{ and }x > 0\text{, then }cx \geq x\text{.} 
\end{equation*}
In other words, $x$ needs to be a positive number.
Here's proof:
\begin{flalign*} 
	& & &\text{Assume }c \geq 1 \text{ and } x>0, & & \llap{}
	\\
	& & &\text{Then, }c\text{ and }x\text{ are positive numbers.} & & \llap{Axiom 1A}
	\\
	& & &\text{Also, a product of positive numbers is positive.} & & \llap{Axiom 1C}
	\\
	& & &\text{Therefore, }c \cdot x\text{ is a poistive number.} & & \llap{}
	\\
	& & &\text{Then, }c \geq 1, & & \llap{By assumption}
	\\
	& & &c \cdot x \geq 1 \cdot x & & \llap{Theorem 14A}
	\\
	& & &\leftrightarrow cx \geq x \;\;\; \qedsymbol & & \llap{}
\end{flalign*}

\hfill
\newpage

% ----------------------------- %
% Section 6
% ----------------------------- %
\bgroup
\color{abs}
\hrule
\egroup

\begin{abstract}
	B27. Resolve this conjecture:
	
	\textbf{
		\\
		If, for all $\epsilon >  0$, $x < c + \epsilon$,
		then $x < c$.
	}
	\noindent\textbf{\\}
\end{abstract}

\bgroup
\color{abs}
\hrule
\egroup

\section*{Counterexample}

This is false. The following counterexample shows
a case in which the hypothesis is true but the conclusion is false.
\begin{flalign*} 
	& & &\text{Let }x=1,c=1, & & \llap{}
	\\
	& & &\text{Let }\epsilon > 0, & & \llap{By hypothesis}
	\\
	& & &\text{Then }x=1<1+\epsilon = c+\epsilon \text{, but }x=1 \geq 1=c & & \llap{}
\end{flalign*}