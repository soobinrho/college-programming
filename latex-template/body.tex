
% ----------------------------- %
% Section 1
% ----------------------------- %
\section*{B3. $Conjecture: \;\; If \;\; for \;\; all \;\; \epsilon>0, \; x<c+\epsilon, \; then \;\; x \leq c.$}
This can be proven with proof by contrapositive
$\forall x, c \in \mathbb{R}, \;\; x>c \Rightarrow \exists \; \epsilon >0 \;\; s.t. \;\; x \geq c+ \epsilon $
\begin{flalign*} 
	&& &Let \; x,c \in \mathbb{R}, \; where \;\; x>c && \llap{}
	\\
	&& &Choose \;\; \epsilon =x-c && \llap{}
	\\
	&& &Since \;\; x>c, \; x-c>0 && \llap{Theorem 3.1.3}
	\\
	&& &Also, \; c+ \epsilon =c+(x-c)=x && \llap{}
	\\
	&& &Therefore, \; c+ \epsilon \leq x && \llap{\qedsymbol}
\end{flalign*}
\hfill

\section*{B4. $Prove: \;\; There \;\; are \;\; no \;\; integers \;\; a \;\; and \;\; b \;\; such \;\; that \;\; b^{2}=4a+2.$}
\begin{flalign*} 
	&& &Filler && \llap{}
	\\
	&& &Filler && \llap{}
	\\
	&& &Filler && \llap{}
	\\
	&& &Filler && \llap{}
	\\
	&& &Filler && \llap{}
	\\
	&& &Filler && \llap{\qedsymbol}
\end{flalign*}
\hfill

\section*{B6. $Prove: \;\; If \;\; n^{3} \;\; is \;\; even, \; so \;\; is \;\; n.$}
\begin{flalign*} 
	&& &Filler && \llap{}
	\\
	&& &Filler && \llap{}
	\\
	&& &Filler && \llap{}
	\\
	&& &Filler && \llap{}
	\\
	&& &Filler && \llap{}
	\\
	&& &Filler && \llap{\qedsymbol}
\end{flalign*}
\hfill

\section*{B7. $Prove: \;\; Let \;\; a \;\; and \;\; b \;\; be \;\; real \;\; numbers. \;\; If \;\; a>0 \;\; and \;\; b>0, \; then \;\; \frac{2}{a}+\frac{2}{b} \neq \frac{4}{a+b}$}
\begin{flalign*} 
	&& &Filler && \llap{}
	\\
	&& &Filler && \llap{}
	\\
	&& &Filler && \llap{}
	\\
	&& &Filler && \llap{}
	\\
	&& &Filler && \llap{}
	\\
	&& &Filler && \llap{\qedsymbol}
\end{flalign*}
\hfill

\section*{B9. $Prove: \;\; \sqrt{3} \;\; is \;\; not \;\; a \;\; rational \;\; number.$}
\begin{flalign*} 
	&& &Filler && \llap{}
	\\
	&& &Filler && \llap{}
	\\
	&& &Filler && \llap{}
	\\
	&& &Filler && \llap{}
	\\
	&& &Filler && \llap{}
	\\
	&& &Filler && \llap{\qedsymbol}
\end{flalign*}
\hfill



\hfill
% ----------------------------- %
% Section eg
% ----------------------------- %
\bgroup
\color{abs}
\hrule
\egroup

\begin{abstract}
	B8-B14. Resolve the conjecture of the same number.
	
	\noindent\textbf{}
	
\end{abstract}
\bgroup
\color{abs}
\hrule
\egroup

\begin{flalign*} 
	&& &Let \;\; x \in \mathbb{R} \;\; where \;\; x \neq 0 && \llap{}
	\\
	&& &Choose \;\; \delta = \frac{|x|}{2} && \llap{}
	\\
	&& &Since \;\; x \neq 0, \; |x|>0 && \llap{Axiom 3.1.1A, Thorem 3.2.2A}
	\\
	&& &Since \;\; \frac{1}{2}>0, \; \epsilon=\frac{|x|}{2}>0 && \llap{Axiom 3.1.1C}
	\\
	&& &Since \;\; \frac{1}{2} \leq 1 \;\; and \;\; |x|>0,&& \llap{Theorem 3.1.13A}
	\\
	&& &\epsilon = \frac{|x|}{2} \leq |x| && \llap{\qedsymbol}
\end{flalign*}

\section*{Section 3.2 B3. $With \;\; the \;\; list \;\; approach, \;\; a \;\; statement \;\; is \;\; not \;\; regarded \;\; as \;\; true \;\; until \\ it \;\; is \;\; on \;\; the \;\; list. \;\; With \;\; this \;\; approach, \;\; when \;\; can \;\; a \;\; statement \;\; be \;\; regarded \;\; as \;\; fales?$}

