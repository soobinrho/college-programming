
% ----------------------------- %
% Section 1
% ----------------------------- %
\section*{B4. $Prove: \;\; 2+2^2+2^3+...+2^n=2^{n+1}-2, \; for \;\; all \;\; n.$}



\hfill
\section*{B9. $Let \;\; x_1=0 \;\; and, \; for \;\; all \;\; n, \; x_{n+1}=\sqrt{2+x_n}$}
This is a sequence. The question tells us what the first term is, and shows us how to get to the next term. As instructed, I solved this problem by first proving $\sqrt{x} \;\; is \;\; an \;\; increasing \;\; function$, which can be translated into more primitive terms ``$\forall \; x,y \in domain \;\; of \;\; square \;\; root \;\; function, \; x<y \Rightarrow \sqrt{x}\leq\sqrt{y}$ ''. This, in turn, is equivalent to ``$\forall \; x,y \in domain \;\; of \;\; sqrt, \; \sqrt{x}>\sqrt{y}\Rightarrow x > y$ ''. because of contrapositive.
\begin{flalign*} 
	&& &Let \;\; x,y \in domain \;\; of \;\; square \;\; root \;\; function.&& \llap{}
	\\
	&& &Assume \;\; \sqrt{y}<\sqrt{x}&& \llap{(By hypothesis)}
	\\
	&& &Since \;\; 0<a<b \Rightarrow a^2 < b^2,&& \llap{(Theorem 3.1.17)}
	\\
	&& &By \;\; definition \;\; of \;\; square \;\; root, \; 0 \leq \sqrt{y} < \sqrt{x}&& \llap{}
	\\
	&& \Rightarrow \; &y=(\sqrt{y})^2 < (\sqrt{x})^2=x && \llap{(Theorem 3.1.17)}
	\\
	&& \Rightarrow \; &y<x && \llap{(Theorem 3.1.17)}
	\\
	&& &Thus, \; \forall \; x,y \in domain \;\; of \;\; sqrt, \; \sqrt{x}>\sqrt{y}\Rightarrow x > y&& \llap{\qedsymbol}
\end{flalign*}

\section*{B9a. $Prove: \;\; x_n <2, \; for \;\; all \;\; n.$}
\begin{flalign*} 
	&& &Base \;\; Case: \;\; Let \;\; n=1 && \llap{}
	\\
	&& &Then, \; x_1 = 0<2 && \llap{}
	\\
	\\
	&& &Induction \;\; Step: \;\; Assume \;\; 
	x_n <2 && \llap{}
	\\
	&& &x_n <2 \Rightarrow 2+x_n < 4&& \llap{}
	\\
	&& \Rightarrow \; & \sqrt{2+x_n}<\sqrt{4} && \llap{}
	\\
	&& \Rightarrow \; & x_{n+1}<2 && \llap{(By substitution)}
	\\
	\\
	&& &By \;\; induction, \; x_n < 2, \; \forall \; n \in \mathbb{N}&& \llap{\qedsymbol}
\end{flalign*}

\section*{B9b. $Prove: \;\; x_{n+1} <x_n , \; for \;\; all \;\; n.$}
\begin{flalign*} 
	&& &Base \;\; Case: \;\; Let \;\; n=1 && \llap{}
	\\
	&& \Rightarrow \; & x_1 =0 && \llap{}
	\\
	&& \Rightarrow \; & x_{1+1}=\sqrt{2+x_1}=\sqrt{2+0}=\sqrt{2}>0=x_1 && \llap{}
	\\
	\\
	&& &Induction \;\; Step: \;\; x_{n+1}>x_n && \llap{}
	\\
	&& &x_{n+1}>x_n \Rightarrow 2+x_{n+1}>2+x_n&& \llap{(Theorem 3.1.4)}
	\\
	&& \Rightarrow \; & \sqrt{2+x_{n+1}}>\sqrt{2+x_n} && \llap{(by claim)}
	\\
	&& \Rightarrow \; & x_{n+2}>x_{n+1} && \llap{(By substitution)}
	\\
	\\
	&& &By \;\; induction, \; x_{n+1} > x_n, \; \forall \; n \in \mathbb{N}&& \llap{\qedsymbol}
\end{flalign*}
\hfill

% ----------------------------- %
% Section 2
% ----------------------------- %
\bgroup
\color{abs}
\hrule
\egroup

\begin{abstract}
	B13-B19. For the next problems, all variables are integers. Definition: $n$ is divisible by m (or, m divides n) iff there exists an integer $k$ such that $n=km$.
	
	\noindent\textbf{}
	
\end{abstract}

\bgroup
\color{abs}
\hrule
\egroup

\section*{B13. $Prove \;\; by \;\; induction: \;\; 4^n-1 \;\; is \;\; divisible \;\; by \;\; 3, \; for \;\; all \;\; n.$}




\hfill
\section*{B18. $Prove \;\; n^3+5n+6 \;\; is \;\; divisible \;\; by \;\; 3, \; for \;\; all \;\; n.$}





\hfill

% ----------------------------- %
% Section 3
% ----------------------------- %
\bgroup
\color{abs}
\hrule
\egroup

\begin{abstract}
	B20-B21. In the next two, $a$ and $b$ are digits from $0$ to $9$.
	
	\noindent\textbf{}
	
\end{abstract}

\bgroup
\color{abs}
\hrule
\egroup

\section*{B20. $Prove: \;\; If \;\; a+b=9, \; then \;\; 9 \;\; divides \;\; a(10^n)+b, \; for \;\; all \;\; n.$}
\begin{flalign*} 
	&& &Base \;\; Case: \;\; Let \;\; n=1 \;\; where \;\; a,b \in \{0, 1, ..., 9\} && \llap{}
	\\
	&& \Rightarrow \; & a \cdot 10 + b=9a+(a+b)=9a+(9)=9(a+1) && \llap{}
	\\
	&& \Rightarrow \; & a+1 \in \mathbb{Z} && \llap{$(Since \;\; a,b \in \{0, 1, ..., 9\})$}
	\\
	&& \Rightarrow \; & 9(a+1) \;\; is \;\; divisible \;\; by \;\; 9 && \llap{}
	\\
	\\
	&& &Induction \;\; Step: \;\; Assume \;\; a(10^n)+b \;\; is \;\; divisible \;\; by \;\; 9 && \llap{}
	\\
	&& \Rightarrow \; & \exists \; k \in \mathbb{Z} \;\; s.t. \;\; a(10^n)+b=9k && \llap{(Definition of divisible)}
	\\
	&&& a(10^{n+1})+b=10\cdot a \cdot 10^n + b && \llap{}
	\\
	&& & = 10(a \cdot 10^n +b) -10b +b && \llap{}
	\\
	&& & = 10(9k) -9b && \llap{}
	\\
	&& & = 9(10k-b) && \llap{}
	\\
	&& & Since \;\; k \in \mathbb{Z} \;\; and \;\; b \in \mathbb{Z}, \; 10k-b \in \mathbb{Z} && \llap{}
	\\
	&& \Rightarrow \; & a(10^{n+1})+b \;\; is \;\; divisible \;\; by \;\; 9  && \llap{(Definition of divisible)}	
	\\
	\\
	&& &By \;\; induction, \; \forall \; n \in \mathbb{N}, \; a(10^n)+b \;\; is \;\; divisible \;\; by \;\; 9&& \llap{\qedsymbol}
\end{flalign*}
\hfill

\section*{B35. $[Calculus] \;\; Prove, \; using \;\; induction \;\; and \;\; the \;\; Product \;\; Rule \;\; and \;\; the \;\; derivative \;\; \\ of \;\; x \;\; is \;\; 1, \;\; that \;\; the \;\; derivative \;\; of \;\; f(x)=x^n \;\; is \;\; f'(x)=nx^{n-1} \;\; for \;\; integer \;\; n \geq 1.$}
First of all, I can rewrite the question as
$\forall \; n \in \mathbb{N}, \; \frac{d}{dx}(x^n)=nx^{n-1}$.
Also, as prior results, we know $\frac{d}{dx}(x)=1$ and $\frac{d}{dx}(f(x)g(x))=f(x)g'(x)+f'(x)g(x)$.
\begin{flalign*} 
	&& &Base \;\; Case: \;\; Let \;\; n=1 && \llap{}
	\\
	&& &Then, \; \frac{d}{dx}(x^1)=1=(1)x^{1-1} && \llap{}
	\\
	\\
	&& &Induction \;\; Step: \;\; Assume \;\; \frac{d}{dx}(x^n)=nx^{n-1} && \llap{}
	\\
	&& &\frac{d}{dx}(x^{n+1})=\frac{d}{dx}(x^n x^1)=x \cdot \frac{d}{dx}(x^n)+\frac{d}{dx}(x) \cdot x^n&& \llap{}
	\\
	&& &=x(nx^{n-1})+(1)x^n && \llap{}
	\\
	&& &=nx^n + x^n = (n+1)x^n && \llap{}
	\\
	\\
	&& &By \;\; induction, \; \forall \; n \in \mathbb{N}, \; \frac{d}{dx}(x^n)=nx^{n-1}&& \llap{\qedsymbol}
\end{flalign*}
\hfill