
% ----------------------------- %
% Section 1
% ----------------------------- %
\section*{Solution}
The first thing that comes to my mind
when I look at this problem is that
the denominator can be factored into
$ (\theta + 3)(\theta + 2) $ which
means our problem becomes
\begin{equation}
	\textbf{33.}\quad \int_{-1}^{\infty} 
	\frac{d\theta}{\theta^2+5\theta+6}
	=
	\int_{-1}^{\infty} 
	\frac{d\theta}{(\theta + 3)(\theta + 2)}	
\end{equation}

After that, we can use partial fraction decomposition
to make the integrand easier to integrate.
\begin{equation}
	\int_{-1}^{\infty} 
	\frac{d\theta}{(\theta + 3)(\theta + 2)}
	=
	\int_{-1}^{\infty} 
	\frac{A}{\theta + 3} + \frac{B}{\theta + 2} d\theta
\end{equation}

We can then multiply both sides of the
equation with $ (\theta + 3)(\theta + 2) $.
For the sake of readability, let's 
skip the integral notation for now
and look at only the coefficients now.
\begin{align}
	1
	&=
	A(\theta + 2) + B(\theta + 3) \\
	(0)\theta + (1)
	&=
	(A+B)\theta + (3A + 2B)
\end{align}

Solving for these two equations
$ A + B = 0 $ and 
$ 3A + 2B = 1 $ gives us
\begin{align}
	A &= 1 \\
	B &= -1
\end{align}

In other words, if we put everything 
back to the original equation together,
we get something that's much more easier
to integrate: 
\begin{equation*}
	\int_{-1}^{\infty} 
	\frac{d\theta}{(\theta + 3)(\theta + 2)}
	=
	\int_{-1}^{\infty} 
	\frac{1}{\theta + 3} + \frac{-1}{\theta + 2} d\theta
\end{equation*}

\newpage

Well, why is it easier to integrate?
It's because 
$ \int \frac{1}{x} $ is elegantly
$ ln|x| + C $.
\begin{align}
	\int_{-1}^{\infty} 
	\frac{d\theta}{(\theta + 3)(\theta + 2)}
	&=
	\Bigg[
	  ln|\theta + 2| - ln|\theta + 3|
	\Bigg]_{-1}^{\infty} \\
	&=
	ln2
\end{align}

Sorry! I skipped over an important
part of how this works. If you look at
the equation above, we have an integral
with infinity. Specifically,
the function we have here is what
we call Type 1 improper integral.
By the way, there are two types of
improper integrals. 

Type 1 improper
integral refers to integrals like
ours, where one or both of the bounds
have infinity as in $ \int_{-1}^{\infty} $

Type 2 improper integral, on the other hand,
refer to integrals whose bounds
have a discontinuity between them.
Anyway, our function doesn't have any
discontinuity, so let's see how to
evaluate a Type 1 improper integral.

First of all, from Lesson 16
\textit{Integrals with the infinite},
we learned that
\begin{equation}
    \int_{-1}^{\infty} f(\theta) =
	\Big[    
	  \lim_{d \to \infty} f(\theta)
	\Big]
	- 
	\Big[ 
	  f(-1)
	\Big]
\end{equation}

Therefore, our problem is equal to
\begin{align}
	&= 
	\Big[ 
	  \lim_{d \to \infty} ln|d + 2| - ln|d + 3|
	\Big]
	- 
	\Big[ 
	  ln|(-1) + 2| - ln|(-1) + 3|
	\Big] \\
	&=
	\Big[ 
      \lim_{d \to \infty} 
      ln\frac{|d + 2|}{|d + 3|}
    \Big]
    - 
    \Big[ 
      ln|1| - ln|2|
    \Big] \\
	&=
	\Big[ 
      \lim_{d \to \infty} 
      ln\frac{d + 2}{d + 3}
    \Big]
    - 
    \Big[ 
      - ln2
    \Big]
\end{align}

Now, we can use L'Hôpital's rule that
we learned from Lesson 15
\textit{Limits with the infinite}.
\begin{align}
	\Big[
      \lim_{d \to \infty} 
	  ln\frac{d + 2}{d + 3} 
	\Big]
	&=
	\Big[
	  \lim_{d \to \infty} 
	  ln\frac{d + 2}{d + 3} \cdot
	  \frac{\frac{d}{d}}{\frac{d}{d}} 
	\Big] \\
	&=
	\Big[
	  \lim_{d \to \infty}
	  ln\frac{1+\frac{2}{d}}{1+\frac{3}{d}} 
	\Big] \\
	&=
	ln
	\frac{
	  \lim_{d \to \infty} 	
	  \left[
	      1+\frac{2}{d}
	  \right]
	}{
	  \lim_{d \to \infty} 
	  \left[
	    1+\frac{3}{d} 
	  \right]
	} \\
    &=
    ln
    \frac{1}{1} \\
    &= 0
\end{align}

Putting this back to the original problem,
we finally get
\begin{equation}
	\textbf{33.}\quad \int_{-1}^{\infty} 
	\frac{d\theta}{\theta^2+5\theta+6}
	=
	\Big[0\Big]
	- 
	\Big[ -ln2 \Big]	
	= ln2
\end{equation}

% ----------------------------- %
% Section 2
% ----------------------------- %
\newpage
\section*{Verifying the Integral}

Let's verify our integral by 
differentiating the integrand and
see if both are equal to each other.

On Maple, let's type the following,
which basically means differentiate
$ ln|\theta + 2| - ln|\theta + 3| $ 
which is the integral we got:
$ \text{diff} \Big(
  \text{ln}(
    \text{abs}(
      \theta + 2
    )
  )
  -
  \text{ln}(
    \text{abs}(
      \theta + 3
    )
  )
\Big) $ 
Then, the output is the same as the
integrand of our original question,
which is 
$ \frac{1}{\theta^2+5\theta+6} $

Therefore, we can finally confirm
that at least the indefinite version
of our answer is a correct answer
to our problem. To be honest,
however,  I think there must be a
better way of verifying our answer.
One thought I came across was to use
Simpson's Rule - which we learned
in June, 2022 - to approximate
the definite integral of
$ \textbf{33.}\quad \int_{-1}^{\infty} 
\frac{d\theta}{\theta^2+5\theta+6} $
However, it has infinity as a boundary.

So, I couldn't come up with
another way of verifying our answer,
but I'll ask this question on
our Unit 4 Forum as a discussion :)
Thank you for reading my AFP, and
hope to see you again for AFP5!
